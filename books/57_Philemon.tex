\bookheader{Philemon}
\labelbook{Phlm}

\bookpretitle{The Letter from Paul to}
\booktitle{Philemon}

\passage{Greetings}

\v{1}From:\fnote{The Gk. lacks \fbib{From}} Paul, a prisoner of the Messiah\fnote{Or \fbib{Christ}} Jesus, and Timothy our brother.

To: Philemon our dear friend\fnote{Or \fbib{our beloved Philemon}} and fellow worker, \v{2}to Apphia our sister, to Archippus our fellow soldier, and to the church in your house.

\v{3}May grace and peace from God our Father and the Lord Jesus, the Messiah,\fnote{Or \fbib{Christ}} be yours!\fnote{The Gk. \fbib{yours} is pl.}
\passage{Paul's Prayer for Philemon}

\v{4}I always thank my God when I mention you\fnote{From verse 4 through v. 21, \fbib{you} and \fbib{your} are sing.} in my prayers, \v{5}because I keep hearing about your love for all the saints and the faith that you have in the Lord Jesus. \v{6}I pray\fnote{The Gk. lacks \fbib{I pray}} that your partnership in the faith may become effective as you fully acknowledge every blessing that is ours\fnote{Other mss. read \fbib{yours} (pl.)} in the Messiah.\fnote{Or \fbib{Christ}} \v{7}For I have received considerable joy and encouragement from your love, because the hearts of the saints have been refreshed, brother, through you.
\passage{Paul's Plea for Onesimus}

\v{8}For this reason, although in the Messiah\fnote{Or \fbib{Christ}} I have complete freedom to order you to do what is proper, \v{9}I prefer to make my appeal on the basis of love. I, Paul, as an old man and now a prisoner of the Messiah\fnote{Or \fbib{Christ}} Jesus, \v{10}appeal to you on behalf of my child Onesimus, whose father I have become during my imprisonment. \v{11}Once he was useless to you, but now he is very useful\fnote{The Gk. name \fbib{Onesimus} means \fbib{useful}} both to you and to me. \v{12}As I send him back, it's like I'm coming along with him.\fnote{Lit. \fbib{back, it's with my innards}} \v{13}I wanted to keep him with me so that he could serve me in your place during my imprisonment for the gospel. \v{14}Yet I did not want to do anything without your consent, so that your good deed might not be something forced, but voluntary. \v{15}Perhaps this is why he was separated from you for a while, so that you could have him back forever, \v{16}no longer as a slave but better than a slave---as a dear brother, especially to me, but even more so to you, both as a person and as a believer.\fnote{Or \fbib{both in the flesh and in the Lord}}

\v{17}So if you consider me a partner, welcome him as you would welcome\fnote{The Gk. lacks \fbib{you would welcome}} me. \v{18}If he has wronged you in any way or owes you anything, charge it to my account. \v{19}I, Paul, am writing this with my own hand: I will repay it. (I will not mention to you that you owe me your very life.) \v{20}Yes, brother, I desire this favor from you in the Lord. Refresh my heart in the Messiah!\fnote{Or \fbib{Christ}} \v{21}Confident of your obedience, I am writing to you because I know that you will do even more than I ask. \v{22}Meanwhile, prepare a guest room for me, too, for I am hoping through your prayers to be returned to you.
\passage{Greetings from Paul's Fellow Workers}

\v{23}Epaphras, my fellow prisoner in the Messiah\fnote{Or \fbib{Christ}} Jesus, sends you\fnote{The Gk. \fbib{you} is sing.} greetings, \v{24}as do Mark, Aristarchus, Demas, and Luke, my fellow workers. \v{25}May the grace of our\fnote{Other mss. read \fbib{the}} Lord Jesus, the Messiah,\fnote{Or \fbib{Christ}} be with your spirit! Amen.\fnote{Other mss. lack \fbib{Amen.}}
