\bookheader{Philippians}
\labelbook{Phil}

\bookpretitle{The Letter from Paul to the}
\booktitle{Philippians}

\labelchapt{1}
\passage{Greetings from Paul and Timothy}

\chapt{1}
\v{1}From:\fnote{The Gk. lacks \fbib{From}} Paul and Timothy, servants of the Messiah\fnote{Or \fbib{Christ}} Jesus.

To: All the holy ones\fnote{Or \fbib{saints}} in Philippi, along with their overseers\fnote{Or \fbib{bishops}} and ministers,\fnote{Or \fbib{deacons}} who are in union with the Messiah\fnote{Or \fbib{Christ}} Jesus.

\v{2}May grace and peace from God our Father and the Lord Jesus, the Messiah,\fnote{Or \fbib{Christ}} be yours!
\passage{Paul's Thanksgiving and Prayer for the Philippians}

\v{3}I thank my God every time I remember you,\fnote{Or \fbib{every time you remember me}} \v{4}always praying joyfully in every one of my prayers for all of you \v{5}because of your partnership in the gospel from the first day until now. \v{6}I am convinced of this, that the one who began a good action among\fnote{Or \fbib{in}} you will bring it to completion by the Day of the Messiah\fnote{Or \fbib{Christ}} Jesus. \v{7}For it is only right for me to think this way about all of you, because you're constantly on my mind.\fnote{Lit. \fbib{because I have you in my heart}; or \fbib{you have me in your heart}} Both in my imprisonment and in the defense and confirmation of the gospel, all of you are partners with me in this privilege.\fnote{Or \fbib{grace}} \v{8}For God is my witness how much I long for all of you with the compassion that the Messiah\fnote{Or \fbib{Christ}} Jesus provides.

\v{9}And this is my prayer, that your love will keep on growing more and more with full knowledge and insight, \v{10}so that you may be able to choose what is best and be pure and blameless until the day when the Messiah\fnote{Or \fbib{Christ}} returns, \v{11}having been filled with the fruit of righteousness that comes through Jesus the Messiah\fnote{Or \fbib{Christ}} so that God will be glorified and praised.
\passage{The Priority of the Gospel in Everything}

\v{12}Now I want you to know, brothers, that what has happened to me has actually caused the gospel to advance. \v{13}As a result, it has become clear to the whole imperial guard and to everyone else that I am in prison for preaching about\fnote{The Gk. lacks \fbib{preaching about}} the Messiah.\fnote{Or \fbib{Christ}} \v{14}Moreover, because of my imprisonment the Lord has caused most of the brothers to become confident to speak God's word more boldly and courageously than ever before. \v{15}Some are preaching the Messiah\fnote{Or \fbib{Christ}} because of jealousy and dissension, while others do so\fnote{The Gk. lacks \fbib{do so}} because of their good will. \v{16}The latter are motivated\fnote{The Gk. lacks \fbib{are motivated}} by love, because they know that I have been appointed to defend the gospel. \v{17}The former proclaim the Messiah\fnote{Or \fbib{Christ}} because they are selfishly ambitious and insincere, thinking that they will stir up trouble for me during my imprisonment.

\v{18}But so what? Just this---that in every way, whether by false or true motives, the Messiah\fnote{Or \fbib{Christ}} is being proclaimed. Because of this, I rejoice and will continue to rejoice. \v{19}I know that this will result in my deliverance through your prayers and the help that comes from the Spirit of Jesus the Messiah.\fnote{Or \fbib{Christ}} \v{20}I rejoice because I eagerly expect and hope that I will have nothing to be ashamed of, because through my\fnote{Lit. \fbib{with all}} boldness the Messiah\fnote{Or \fbib{Christ}} will be exalted through me,\fnote{Lit. \fbib{exalted in my body}} now as always, whether I live or die.\fnote{Lit. \fbib{by life or by death}}

\v{21}For to me, to go on living is the Messiah,\fnote{Or \fbib{Christ}} and to die is gain. \v{22}Now if I continue living, fruitful labor is the result, so I do not know which I would prefer. \v{23}Indeed, I cannot decide between the two. I have the desire to leave this life and be with the Messiah,\fnote{Or \fbib{Christ}} for that is far better. \v{24}But for your sake it is better that I remain alive.\fnote{Lit. \fbib{remain in this body}}

\v{25}Since I am convinced of this, I know that I will continue to live and be with all of you, so you will mature in the faith and know joy in it. \v{26}Then your rejoicing in the Messiah\fnote{Or \fbib{Christ}} Jesus will increase along with mine\fnote{Lit. \fbib{in me}} when I visit with you again.
\passage{Standing Firm in One Spirit}

\v{27}The only thing that matters is that you continue to live as good citizens in a manner worthy of the gospel of the Messiah.\fnote{Or \fbib{Christ}} Then, whether I come to see you or whether I stay away, I may hear all about you---that you are standing firm in one spirit, struggling with one mind for the faith of the gospel, \v{28}and that you are not intimidated by your opponents in any way. This is evidence that they will be destroyed and that you will be saved---and all because of\fnote{Lit. \fbib{and that from}} God. \v{29}For you have been given the privilege\fnote{Lit. \fbib{it has been given you}} for the Messiah's\fnote{Or \fbib{Christ's}} sake not only to believe in him but also to suffer for him. \v{30}You have the same struggle that you saw in me and now hear that I am still having.\fnote{Lit. \fbib{hear in me}}
\labelchapt{2}
\passage{Unity through Humility}

\chapt{2}
\v{1}Therefore, if there is any encouragement in the Messiah,\fnote{Or \fbib{Christ}} if there is any comfort of love, if there is any fellowship in the Spirit, if there is any compassion and sympathy, \v{2}then fill me with joy by having the same attitude, sharing the same love, being united in spirit, and keeping one purpose in mind. \v{3}Do not act out of selfish ambition or conceit, but with humility think of others as being better than yourselves. \v{4}Do not be concerned about your own interests, but also be concerned about\fnote{The Gk. lacks \fbib{be concerned about}} the interests of others. \v{5}Have the same attitude among yourselves\fnote{Or \fbib{Have this mind in you}} that was also in the Messiah\fnote{Or \fbib{Christ}} Jesus:\fnote{Verses 6-11 probably represent an early Christian hymn.}

\begin{poetry}
\poeml \v{6}In God's own form existed he, \\
\poemll    and shared with God equality, \\
\poemlll       deemed nothing needed grasping. \\
\poeml \v{7}Instead, poured out in emptiness, \\
\poemll    a servant's form did he possess, \\
\poemlll       a mortal man becoming. \\
\poeml In human form he chose to be, \\
\poeml \v{8}and lived in all humility, \\
\poemlll       death on a cross obeying. \\
\poeml \v{9}Now lifted up by God to heaven, \\
\poemll    a name above all others given, \\
\poemlll       this matchless name possessing. \\
\poeml \v{10}And so, when Jesus' name is called, \\
\poemll    the knees of everyone should fall,\fnote{Or \fbib{every knee should bend}} \\
\poemlll       wherever they're residing.\fnote{Lit. \fbib{in heaven, on earth, and under the earth}} \\
\poeml \v{11}Then every tongue in one accord, \\
\poemll    will say that Jesus the Messiah\fnote{Or \fbib{Christ}} is Lord, \\
\poemlll       while God the Father praising.
\end{poetry}
\passage{Blameless Living}

\v{12}And so, my dear friends, just as you have always obeyed, not only when I was with you but even more now that I am absent, continue to work out your salvation with fear and trembling. \v{13}For it is God who is producing in you both the desire and the ability to do what pleases him. \v{14}Do everything without complaining or arguing \v{15}so that you may be blameless and innocent, God's children without any faults among a crooked and perverse generation, among whom you shine like stars in the world \v{16}as you hold firmly to the word of life. Then I will be proud when the Messiah\fnote{Or \fbib{Christ}} returns\fnote{Lit. \fbib{will boast in the day of the Messiah}} that I did not run in vain or work hard in vain.

\v{17}Yet even if I am being poured out like an offering as part of the sacrifice and service I offer\fnote{The Gk. lacks \fbib{I offer}} for your faith, I rejoice, and I share my joy with all of you. \v{18}In the same way, you also should rejoice and share your joy with me.
\passage{News about Paul's Companions}

\v{19}Now I hope in the Lord Jesus to send Timothy to you soon so that I can be encouraged when I learn of your condition. \v{20}I do not have anyone else like him who takes a genuine interest in your welfare. \v{21}For all the others look after their own interests, not after those of Jesus the Messiah.\fnote{Or \fbib{Christ}} \v{22}But you know his proven worth---how like a son with his father he served with me in the gospel. \v{23}Therefore, I hope to send him as soon as I see how things are going to turn out for me. \v{24}Indeed, I am confident in the Lord that I will come to visit you\fnote{The Gk. lacks \fbib{to visit you}} soon.

\v{25}Meanwhile, I thought it best to send Epaphroditus---my brother, fellow worker, and fellow soldier, but your messenger and minister to my need---back to you. \v{26}For he has been longing for\fnote{Other mss. read \fbib{longing to see}} all of you and is troubled because you heard that he was sick. \v{27}Indeed, he was sick to the point of death, but God had mercy on him, and not only on him but also on me, so that I would not have one sorrow on top of another.\fnote{Lit. \fbib{sorrow on sorrow}} \v{28}Therefore, I am especially eager to send him so that you may have the joy of seeing him again, and so that I may feel relieved. \v{29}So joyfully welcome him in the Lord and make sure you honor such people highly, \v{30}because he came close to death for the work of the Messiah\fnote{Or \fbib{Christ}; other mss. read \fbib{Lord}} by risking his life to complete what remained unfinished in your service to me.
\labelchapt{3}
\passage{Warning against Pride}

\chapt{3}
\v{1}So then,\fnote{Or \fbib{Furthermore}} my brothers, keep on rejoicing in the Lord. It is no trouble for me to write the same things to you; indeed, it is for your safety.

\v{2}Beware of the dogs! Beware of the evil workers! Beware of the mutilators!\fnote{Lit. \fbib{the mutilation}; Lit. \fbib{katatome} (a cutting off)} \v{3}For it is we who are the circumcision\fnote{Lit. \fbib{peritome} (a cutting around)}---we who worship in the Spirit of God\fnote{Other mss. read \fbib{worship God in the Spirit}} and find our joy in the Messiah\fnote{Or \fbib{Christ}} Jesus. We have not placed any confidence in the flesh, \v{4}although I could have confidence in the flesh. If anyone thinks he can place confidence in the flesh, I have more reason to think so.\fnote{Lit. \fbib{I more}} \v{5}Having been circumcised on the eighth day, I am of the nation of Israel, from the tribe of Benjamin, a Hebrew of Hebrews. As far as the Law is concerned, I was a Pharisee. \v{6}As for my zeal, I was a persecutor of the church. As far as righteousness in the Law is concerned, I was blameless.

\v{7}But whatever things were assets to me, these I now consider a loss for the sake of the Messiah.\fnote{Or \fbib{Christ}} \v{8}What is more, I continue to consider all these things to be a loss for the sake of what is far more valuable, knowing the Messiah\fnote{Or \fbib{Christ}} Jesus, my Lord. It is because of him that I have experienced the loss of all those things. Indeed, I consider them rubbish\fnote{Or \fbib{dung}} in order to gain the Messiah\fnote{Or \fbib{Christ}} \v{9}and be found in him, not having a righteousness of my own that comes from the Law, but one that comes through the faithfulness\fnote{Or \fbib{through faith in}} of the Messiah,\fnote{Or \fbib{Christ}} the righteousness that comes from God and that depends on faith. \v{10}I want to know the Messiah\fnote{Lit. \fbib{To know him}}---what his resurrection power is like and what it means to share in his sufferings by becoming like him in his death, \v{11}though I hope to experience the resurrection from the dead.
\passage{Pursuing the Goal}

\v{12}It's not that I have already reached this goal or have already become perfect. But I keep pursuing it, hoping somehow to embrace it just as I have been embraced by the Messiah\fnote{Or \fbib{Christ}} Jesus. \v{13}Brothers, I do not consider myself to have embraced it yet.\fnote{Other mss. omit \fbib{yet}} But this one thing I do: Forgetting what lies behind and straining forward to what lies ahead, \v{14}I keep pursuing the goal to win the prize\fnote{Lit. \fbib{the goal for the prize}} of God's heavenly call in the Messiah\fnote{Or \fbib{Christ}} Jesus.

\v{15}Therefore, those of us who are mature\fnote{Or \fbib{perfect}} should think this way. And if you think differently about anything, God will show you how to think.\fnote{Lit. \fbib{show you this}} \v{16}However, we should live up to what we have achieved so far.
\passage{True and False Teachers}

\v{17}Join together in imitating me, brothers, and pay close attention to those who live by the example we have set for you.\fnote{Lit. \fbib{the example you have in us}} \v{18}For I have often told you, and now tell you even with tears, that many live as enemies of the cross of the Messiah.\fnote{Or \fbib{Christ}} \v{19}Their destiny is destruction, their god is their belly, and their glory is in their shame. Their minds are set on worldly things.

\v{20}Our citizenship, however, is in heaven, and it is from there that we eagerly wait for a Savior, the Lord Jesus, the Messiah.\fnote{Or \fbib{Christ}} \v{21}He will change our unassuming bodies and make them like his glorious body through the power that enables him to bring everything under his authority.
\labelchapt{4}
\passage{Closing Exhortations}

\chapt{4}
\v{1}Therefore, my dear brothers whom I long for, my joy and my victor's crown, this is how you must stand firm in the Lord, dear friends. \v{2}I urge Euodia and Syntyche to have the same attitude in the Lord. \v{3}Yes, I also ask you, my true partner,\fnote{Or \fbib{my loyal Syzygus}} to help these women. They have worked hard with me to advance\fnote{The Gk. lacks \fbib{to advance}} the gospel, along with Clement and the rest of my fellow workers, whose names are in the Book of Life.

\v{4}Keep on rejoicing in the Lord at all times. I will say it again: Keep on rejoicing! \v{5}Let your gracious attitude\fnote{Lit. \fbib{spirit}} be known to all people. The Lord is near: \v{6}Never worry about anything. Instead, in every situation let your petitions be made known to God through prayers and requests, with thanksgiving. \v{7}Then God's peace, which goes far beyond anything we can imagine, will guard your hearts and minds in union with the Messiah\fnote{Or \fbib{Christ}} Jesus.

\v{8}Finally, brothers, whatever is true, whatever is honorable, whatever is fair, whatever is pure, whatever is acceptable, whatever is commendable, if there is anything of excellence and if there is anything praiseworthy---keep thinking about these things. \v{9}Likewise, keep practicing these things: what you have learned, received, heard, and seen in me. Then the God of peace will be with you.
\passage{The Philippians' Gifts}

\v{10}Now I rejoice in the Lord greatly, because once again you have shown your concern for me. Of course, you were concerned for me but you did not have an opportunity to show it.\fnote{The Gk. lacks \fbib{to show it}} \v{11}I am not saying this because I am in any need, for I have learned to be content in whatever situation I am in. \v{12}I know how to be humble, and I know how to prosper. In each and every situation I have learned the secret of being full and of going hungry, of having too much and of having too little. \v{13}I can do all things through him\fnote{Other mss. read \fbib{the Messiah}} who strengthens me. \v{14}Nevertheless, it was kind of you to share my troubles.

\v{15}You Philippians also know that in the early days\fnote{Lit. \fbib{in the beginning}} of the gospel, when I left Macedonia, no church participated with me in the matter of giving and receiving except for you. \v{16}Even while I was in Thessalonica, you provided for my needs not once, but twice. \v{17}It is not that I am looking for a gift. No, I want to see that you receive the fruit that increases to your benefit. \v{18}I have been paid in full and have more than enough. I am fully supplied, now that I have received from Epaphroditus what you sent---a fragrant aroma, a sacrifice acceptable and pleasing to God. \v{19}And my God will fully supply your every need according to his glorious riches in the Messiah\fnote{Or \fbib{Christ}} Jesus. \v{20}Glory belongs to our God and Father forever and ever! Amen.
\passage{Final Greeting}

\v{21}Greet every saint who is in union with the Messiah\fnote{Or \fbib{Christ}} Jesus. The brothers who are with me send their greetings to you. \v{22}All the saints, especially those of the emperor's\fnote{Or \fbib{Caesar's}} household, greet you.

\v{23}May the grace of the Lord Jesus, the Messiah,\fnote{Or \fbib{Christ}} be with your spirit! Amen.\fnote{Other mss. lack \fbib{Amen}}
